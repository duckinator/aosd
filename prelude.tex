% File soley meant for shared latex macros and imports.
\makeatletter
\DeclareRobustCommand*\cal{\@fontswitch\relax\mathcal}
\usepackage{expl3}
\usepackage{cancel}
\usepackage{fancyvrb}
\usepackage[final]{listings}
\usepackage{multicol}
\usepackage{color}
\usepackage{xcolor}

% following were used in other papers of mine, leaving commented out in
% case I wind up needing one or two of these.
%\usepackage{pdfcolmk}
%\usepackage{supertabular}
%\usepackage{symbolindex}
%\usepackage{amstext}
%\usepackage{amssymb}
%\usepackage{mathtools}
%\usepackage{mathstyle}
%\usepackage{breqn}
%\usepackage{amsmath}
%\usepackage{isomath}
%\usepackage{outline}

\usepackage{xparse}
% Exacting typesetting for SI units.
\usepackage{sistyle}
\usepackage{ifthen}
\usepackage{polynomial}
\usepackage{theoremref}
\usepackage{todonotes}
\usepackage{fancyhdr}
\pagestyle{fancy}
\usepackage[active]{srcltx}
\usepackage{babel}
\usepackage{fixltx2e}
\usepackage{enumitem}
\usepackage{amsthm}
\usepackage{wasysym}
\usepackage[all]{xy}
\PassOptionsToPackage{normalem}{ulem}
\usepackage{ulem}
\usepackage{empheq}
\usepackage{xfrac}
\usepackage{bytefield}
\usepackage{cite}
\usepackage{fixme}


\usepackage{marginnote}

\usepackage{polynom}
\usepackage{exscale}
\usepackage{url}
\usepackage{tikz}

% Import in the whole kitchen sink for TiKz
\usetikzlibrary{arrows,backgrounds,matrix,
  plothandlers,plotmarks,shadings,shadows,
  svg.path,topaths,through,trees,fit,fadings,
  er,circuits,calc,automata,intersections, patterns,positioning,
  decorations.pathmorphing, decorations.pathreplacing,
  decorations.markings, decorations.footprints, decorations.shapes,
  decorations.text, decorations.fractals, chains}

\usepackage[unicode=true,
            bookmarks=true,bookmarksnumbered=false,
            bookmarksopen=true,bookmarksopenlevel=6,
            breaklinks=false,pdfborder={0 0 1},
            backref=page,colorlinks=true] {hyperref}

\usepackage{glossaries}

\setcounter{errorcontextlines}{10}
%\DeclareMathOperator{\rank}{rank}

\newenvironment{sqcases}{%
  \matrix@check\sqcases\env@sqcases
}{%
  \endarray\right.%
}
\def\env@sqcases{%
  \let\@ifnextchar\new@ifnextchar
  \left\lbrack
  \def\arraystretch{1.2}%
  \array{@{}l@{\quad}l@{}}%
}


\providecolor{nix_color_TODO}{rgb}{0.75,0,0}
\providecolor{nix_color_thought}{rgb}{0,0,0.5}
\newcommand{\thought}[1]{{\texorpdfstring{\color{nix_color_thought}{#1}}{}}}

%% Until I find the official one, this denotes LaTeX3.
\newcommand\LaTeXThree{$\LaTeX{}_3$}

\newcommand\keyword[1]{\emph{#1}}
\newcommand{\ba}{\begin{array}}
\newcommand{\ea}{\end{array}}
\newcommand{\bq}{\begin{quote}}
\newcommand{\eq}{\end{quote}}
\newcommand{\bt}{\begin{tabular}}
\newcommand{\et}{\end{tabular}}
\newcommand{\bi}{\begin{itemize}}
\newcommand{\ei}{\end{itemize}}
\newcommand{\be}{\begin{enumerate}}
\newcommand{\ee}{\end{enumerate}}



% For HTML anchors I believe.
\newcommand{\anchor}[2]{#2}
\newcommand{\lb}{[\![}
\newcommand{\rb}{]\!]}
\newcommand{\db}[1]{\lb#1\rb}


\newcommand\TODO[1]{{({\bf $\spadesuit$~TODO:} \texorpdfstring{\color{nix_color_TODO}{\em #1}}{}})}

\newcommand{\indextt}[1]{\index{#1@@{\tt #1}}}
\newcommand{\indexsyn}[1]{\index{#1@@{\it #1}}}
\newcommand{\indexmodule}[1]{\index{#1@@{\tt #1} (module)}}
\newcommand{\indextycon}[1]{\index{#1@@{\tt #1} (datatype)}}
\newcommand{\indexsynonym}[1]{\index{#1@@{\tt #1} (type synonym)}}
\newcommand{\indexclass}[1]{\index{#1@@{\tt #1} (class)}}
\newcommand{\indexdi}[1]{\index{#1@@{\tt #1} (class)!derived instance}}
\newcommand{\indexnote}[1]{#1n}
\newcommand{\emptystr}{[\,]}
\newcommand{\ignorehtml}[1]{#1}
\def\skipline{\vskip 1em}
\newsavebox{\fmbox}


\newcommand{\bkq}{\mbox{\tt \char'022}} % (syntax) backquote char
\newcommand{\bkqB}{\bkq\hspace{-.2em}} % (syntax) backquote char (Before)
\newcommand{\bkqA}{\hspace{-.2em}\bkq}% (syntax) backquote char (After)

\newcommand\sq[1]{[\,#1\,]}

% denotional semantics
\newcommand{\den}[2]{{\cal #1}\db{#2}\,}
\newcommand{\denote}[3]{\[\ba{c} {\cal #1} : #2 \\[1 ex] #3 \ea\]}


% mathy things
\newcommand{\x}{\times}













%%%%%%%%%%%%%%%%%%%%%%%%%%%%%% Textclass specific LaTeX commands.
\numberwithin{section}{chapter}
\numberwithin{equation}{section}
\numberwithin{figure}{section}
\newlength{\lyxlabelwidth}      % auxiliary length 
\numberwithin{table}{section}
\theoremstyle{plain}
\newtheorem{thm}{\protect\theoremname}
  \theoremstyle{definition}
  \newtheorem{example}[thm]{\protect\examplename}
  \theoremstyle{remark}
  \newtheorem{rem}[thm]{\protect\remarkname}
  \theoremstyle{remark}
  \newtheorem{note}[thm]{\protect\notename}
  \theoremstyle{plain}
  \newtheorem{question}[thm]{\protect\questionname}
  \theoremstyle{remark}
  \newtheorem{claim}[thm]{\protect\claimname}
  \theoremstyle{definition}
  \newtheorem*{example*}{\protect\examplename}
  \theoremstyle{remark}
  \newtheorem{notation}[thm]{\protect\notationname}
  \theoremstyle{remark}
  \newtheorem*{note*}{\protect\notename}
  \theoremstyle{definition}
  \newtheorem{defn}[thm]{\protect\definitionname}
  \theoremstyle{plain}
  \newtheorem{fact}[thm]{\protect\factname}
\newcommand{\strong}[1]{\textbf{#1}}
\usepackage{xunicode}
  \providecommand{\claimname}{Claim}
  \providecommand{\definitionname}{Definition}
  \providecommand{\examplename}{Example}
  \providecommand{\notationname}{Notation}
  \providecommand{\notename}{Note}
  \providecommand{\questionname}{Question}
  \providecommand{\remarkname}{Remark}
  \providecommand{\factname}{Fact}
\providecommand{\theoremname}{Theorem}





%\newcommand\Acronym[2]{#2 (#1)}


\ExplSyntaxOn


\DeclareDocumentCommand \Acronym { l m } {
  #2~(#1)
%  \def\NoDocumentation{}
%  \IfNoValueTF {#1} { % Default case, docs at #2.pdf
%    \@pkg{#2}~\pkgdocs{#2}{#2}
%  }{
%    \if_meaning:w #1 \NoDocumentation
%    \@pkg{#2}        % Package has no known documentation.
%    \else:
%    \@pkg{#2}~\pkgdocs{#2}{#1} % Documentation at some other filename.
%    \fi:
%  }
}
\ExplSyntaxOff

\DeclareMathAccent{\ring}{\mathalpha}{operators}{"17}
\providecommand*{\angs}{%
  \ensuremath{\smash{\mathrm{\ring A}}}}

\providecommand*{\ohm}{%
  \ensuremath{\mathrm{\Omega}}}

\providecommand*{\degree}{%
  \ensuremath{^\circ}}

\providecommand*{\celsius}{%
  \ensuremath{\mathrm{^\circ C}}}

\providecommand*{\micro}{%
  \ensuremath{\mu}}

\providecommand*{\unit}[1]{%
  \ensuremath{\mathrm{\,#1}}}

\providecommand*{\ped}[1]{%
  \ensuremath{_\mathrm{#1}}}
\providecommand*{\ap}[1]{%
  \ensuremath{^\mathrm{#1}}}



\makeatother
%%% Local Variables: 
%%% mode: latex
%%% TeX-master: t
%%% End: 
